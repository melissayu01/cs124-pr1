\documentclass[a4paper]{article}

\usepackage{fullpage} % Package to use full page
\usepackage{parskip} % Package to tweak paragraph skipping
\usepackage{tikz} % Package for drawing
\usepackage{amsmath}
\usepackage{hyperref}

\title{Programming Assignment 1 Blah}
\author{HUIDS: 90978217 AND BLAHBLAH}
\date{02/27/2017}

\begin{document}

\maketitle

\section{Introduction}
In this programming assignment, we constructed minimum spanning trees (MSTs) for complete, undirected graphs of 0, 2, and 4 dimension. We then determined how the average total weight of the MST grew as a function of the number of vertices. [PUT IN ABSTRACT?]


\section{Methods}
\subsection{Algorithm Motivation}
We chose to implement the eager implementation of Prim's algorithm. Unlike the lazy implementation of Prim's algorithm, which keeps track of all the edges connecting the MST to any vertex not in the MST, the eager implementation only keeps track of the \textbf{minimum} edges connecting each vertex not in the tree to a vertex in the tree. This way, the number of edges we need to keep track of is $O(n)$, where $n$ is the number of vertices, not $O(n^2)$ as in the Prim's lazy implementation.

Likewise, Kruskal's algorithm also requires $O(|E|)$ space, where $E$ is the set of edges in the graph, since we must keep track of all edges not in the MST, which we thought was less feasible than the eager implementation of Prim's algorithm.

Thus, the eager implementation of Prim's algorithm takes $O(|V|)$ space, whereas the lazy implementation of Prim's and Kruskal's take $O(|E|)$ space. 

\section{Results}
After running our algorithm on MSTs with $n$ vertices, where $n$ is $2^7, 2^8, ..., 2^18$, we obtained the following results:


\section{Discussion}


\end{document}